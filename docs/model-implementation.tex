There are a few different variants of the model used in the paper and
therefore in our project.

For starters, we tried using the GAT model that is bundled with Pytorch
Geometric.
It appears this implementation is not flexible enough to exactly
follow what the paper did, but we tried to stay as close as possible, so we
called it as follows:

\begin{minted}{python}
from torch_geometric.nn import GAT
model = GAT(
    in_channels=dataset.num_features,
    out_channels=dataset.num_classes,
    hidden_channels=8,
    num_layers=2,
    heads=8,
    dropout=0.6,
    act='elu',
    act_first=True
)
\end{minted}



The `hidden\_channels' parameter tells us that the data from each node in the
input graph are projected to an 8-dimensional space via a linear
transformation (a matrix multiplication).
The `act' parameter tells us that
the exponential linear unit is then applied to the transformed data.
`heads' indicates that this is repeated 8 different times, once per ``attention head'',
meaning there are 8 separate linear transformations from the original data's
space to 8-dimensional space.
`dropout' indicates that dropout is applied with
parameter p = 0.6. Lastly, `num\_layers' indicates that what this paragraph
describes is repeated twice, with the output of the first later being fed into
the second layer.

As mentioned, this does not quite follow the models as described in the paper,
but this was a useful step for us to figure out how to feed the data into the
model, and more generally how to set everything up.
Pytorch Geometric implementation of graph attention networks does not allow, for instance, to
specify a different activation function for each layer, which would be
required to exactly follow the paper's methodology.

We therefore aim to build our own graph attention network implementation using
Pytorch Geometric GATConv class, which implements one layer of the GAT
model. Using GATConv directly will allow us to specify different activation functions at each layer and other such details.

We are also attempting to rewrite GAT from scratch, by simply following the
mathematical description of the model from the paper.
Unfortunately, our implementation will likely be a bit lot slower than any of the currently
available implementations on the net as we are still trying to understand, how the graph convolution operation
(e.g.\ Pytorch Geometric GATConv) is equivalent to the mathematical formulae in the paper.
