Before Graph Attention Networks (GATs), several neural network architectures were proposed to process graph-structured data, including Graph Convolutional Networks (GCNs), GraphSAGE, and DeepWalk. However, these architectures had one or more of the following limitations:

\begin{itemize}
    \item Inability to operate on arbitrarily structured graphs: GCNs, for example, are limited to processing homogeneous graphs where all nodes have the same features.

    \item Need to sample from input graphs: some architectures, like GraphSAGE, require sampling from input graphs to create mini-batches, which can lead to information loss.

    \item Learning separate weight matrices for different node degrees: GCNs use different weight matrices for nodes with different degrees, making the model less scalable.

    \item Inability to parallelize across nodes: traditional GCNs perform summation operations over a node's neighbors, which cannot be parallelized.
\end{itemize}

Overall, prior architectures faced difficulties in handling the complexity and heterogeneity of real-world graphs, which motivated the development of Graph Attention Networks.

GATs aim to be able to operate on arbitrarily structured graphs in a manner that is parallelizable across nodes in the graph, thus having none of the
limitations mentioned previously.
GATs use a masked shared self-attention mechanism.
The mask ensures that, for a given node, only features from first-degree neighbours are taken into consideration.
The self-attention mechanism allows the model to assign arbitrary weights to each of a given
node's neighbours, which then allows the neighbours' features to be
combined, which results in a new set of features for the node in question.
The paper only mentions using the resulting features for classification
tasks, but it should also be possible to use these features for regression
tasks.