\documentclass{article}
\usepackage{minted}
\usepackage{setspace}
\usepackage{graphicx}
\usepackage{float}
\usepackage{amsmath}
\usepackage{amsfonts}
\usepackage{amssymb}
\usepackage{microtype}
\usepackage[hyperref]{project}
\usepackage{pythonhighlight}
\usepackage{booktabs,chemformula}
\renewcommand{\UrlFont}{\ttfamily\small}

\aclfinalcopy


\newcommand\BibTeX{B\textsc{ib}\TeX}

\title{Spring 2023 CS598 DL4H: Graph Attention Networks Reproducibility Project}

\author
{Valle-Mena, Ricardo Ruy and Soni, Kushagra \\
    \texttt{\{rrv4, soni14\}@illinois.edu}
    \\[2em]
    Group ID: 179\\
    Paper ID: 7\\
    Presentation link: \url{https://youtu.be/OLJBoOzjBHg} \\
    Code link: \url{https://github.com/kushagrasoni/CS598_DLH_GAT_Implementation}
    \\[2em]
}

\begin{document}
    \maketitle
\section{Introduction}\label{sec:introduction}
    Graph Attention Networks (GATs)~\cite{velickovic2018graph} are a type of neural network architecture designed for
    processing graph-structured data.
    They aim to overcome the limitations of previous architectures by being able to operate on arbitrarily structured graphs in a parallelizable manner.
    GATs use a shared masked self-attention mechanism to assign weights to each node's neighbors and combine their features, resulting in a new set of features for the node in question.
    GATs have shown promising results in various graph-based tasks, including node classification and link prediction.

    \section{Scope of Reproducibility}\label{sec:scope-of-reproducibility}

    \subsection{Background and Problem Statement}\label{sec:background-and-problem-statement}
    Before Graph Attention Networks (GATs), several neural network architectures were proposed to process graph-structured data, including Graph Convolutional Networks (GCNs), GraphSAGE, and DeepWalk. However, these architectures had one or more of the following limitations:

\begin{itemize}
    \item Inability to operate on arbitrarily structured graphs: GCNs, for example, are limited to processing homogeneous graphs where all nodes have the same features.

    \item Need to sample from input graphs: some architectures, like GraphSAGE, require sampling from input graphs to create mini-batches, which can lead to information loss.

    \item Learning separate weight matrices for different node degrees: GCNs use different weight matrices for nodes with different degrees, making the model less scalable.

    \item Inability to parallelize across nodes: traditional GCNs perform summation operations over a node's neighbors, which cannot be parallelized.
\end{itemize}

Overall, prior architectures faced difficulties in handling the complexity and heterogeneity of real-world graphs, which motivated the development of Graph Attention Networks.

GATs aim to be able to operate on arbitrarily structured graphs in a manner
that is parallelizable across nodes in the graph, thus having none of the
limitations mentioned previously. GATs use a masked shared self-attention
mechanism. The mask ensures that, for a given node, only features from
first-degree neighbours are taken into consideration. The self-attention
mechanism allows the model to assign arbitrary weights to each of a given
node's neighbours, which then allows the neighbours' features to be
combined, which results in a new set of features for the node in question.
The paper only mentions using the resulting features for classification
tasks, but it should also be possible to use these features for regression
tasks.

    \subsection{Objectives}\label{sec:objectives}
    Our goal was to reproduce the results reported in the original paper, which used the Cora, Citeseer, Pubmed, and Protein-protein interaction datasets.
    Specifically, we aimed for:
    \begin{itemize}
        \item Classification accuracies of approximately 83.0\%, 72.5\%, and 79\% in the Cora, Citeseer, and Pubmed
        datasets, respectively,
        \item Micro-averaged F1 score of approximately 0.973 in the protein-protein interaction dataset.
    \end{itemize}

    Furthermore, we trained two one-layer GAT models and a three-layer GAT
    model on the Cora, Citeseer and Pubmed datasets as ablation studies.  We
    hypothesized that one-layer models would perform worse than two-layer
    models, which would in turn perform worse than three-layer models.

    \section{Methodology}\label{sec:methodology}
    We tried a total of 5 GAT implementations, four of which were our own, and
    one of which is from the Pytorch Geometric library. One of our
    implementations was strongly influenced by the paper's authors'
    implementation ~\cite{petarvgatgithub}.

    All datasets from the paper are publicly available in multiple locations,
    including the Pytorch Geometric library. We have been running our
    experiments locally. Despite the datasets being relatively small, the PPI
    dataset turned out to be sufficiently large to take a prohibitively long
    time to run. We therefore ended up not having results to present for the
    PPI data.

    \subsection{Model Description}\label{subsec:model-description}
    \subsubsection{Architecture}
There are several variants of the same model used in the paper.
For the Cora and Citeseer datasets, two-layer GAT models were used.  The first
layer has 8 attention heads, projects the input graph's features to an
8-dimensional feature space, and uses an exponential linear unit (ELU) as its
activation function. The second layer has a single attention head, projects
the data to a C-dimensional feature space, where C is the number of classes in
the dataset, and uses a softmax activation function.  L2 regularization is
applied with lambda = 0.0005, and dropout is applied with p = 0.6.

For the Pubmed data, the architecture is mostly the same. However, the second
layer has 8 attention heads like the first layer, and the L2 regularization
uses a coefficient of 0.001 instead of 0.0005.

For the protein-protein interaction data, a three-layer model is used. The
first two layers have 4 attention heads, project their input data to a
256-dimensional feature space, and use an ELU activation function. The third
layer has 6 attention heads, projects its input data to a 121-dimensional
feature space, averages all 121 dimensions, and applies a softmax activation
function.

While training the model for various datasets we employed early stopping strategy similar to the one used in th
original paper, with a patience of 100 epochs.

\subsubsection{Learning Objectives}
The learning objectives for various datasets used are as follows:
\begin{itemize}
    \item Cora: To classify academic papers into one of seven classes based on their content, using citation links
    between the papers as input.
    \item Citeseer: Given a graph where nodes represent research papers and edges represent citation links between
    papers, the model is trained to predict the subject area of each paper based on its citation links and the text
    of its title and abstract.
    The dataset contains 3,327 papers, each belonging to one of six subject areas:
    ``Agents'', ``AI'', ``DB'', ``IR'', ``ML'', or ``HCI''.
    The objective is to correctly classify the papers into their respective subject areas
    \item Pubmed: The goal is to predict the category of a scientific publication based on the citation network of papers.
    There are three possible categories: diabetes mellitus, cardiovascular diseases, and neoplasms
    \item PPI: Given a graph where nodes represent proteins and edges represent interactions between proteins,
    the task is to predict which proteins interact with each other.
\end{itemize}


    \subsection{Data Description}\label{subsec:data-description}
    
In our reproducibility project, we used four citation network datasets: Cora, Citeseer, Pubmed, and Protein-Protein Interaction (PPI).
The Cora, Citeseer and Pubmed datasets were originally introduced in~\cite{sen2008collective} and the PPI dataset was
introduced in~\cite{hamilton2017inductive}.

\subsubsection{Cora}
The Cora dataset consists of 2,708 scientific publications classified into one of seven categories.
The citations between publications form a graph, where each node represents a publication and each edge represents a citation.

\subsubsection{Citeseer}
The Citeseer dataset consists of 3,327 scientific publications classified into one of six categories.
Similarly, the citations between publications form a graph.

\subsubsection{Pubmed}
The Pubmed dataset consists of 19,717 scientific publications from the PubMed database, where each publication is associated with one or more MeSH (Medical Subject Headings) terms.
The graph is constructed using the citation links between publications and each node represents a publication.
The task is to predict the MeSH terms associated with each publication.

\subsubsection{PPI}
The PPI dataset consists of 24 protein-protein interaction networks, where each network is represented by a graph with nodes corresponding to proteins and edges representing interactions between them.
The task is to predict whether a pair of proteins interacts or not.

For all datasets, we used the same pre-processing steps as mentioned in the original GAT paper.
In particular, we normalized the feature vectors for each node and added self-loops to the adjacency matrix.
We also split the datasets into training, validation, and test sets according to the same ratio as in the original paper.

\begin{table}
    \centering
    \begin{tabular}{@{}lllll@{}}
        \toprule
        \textbf{Dataset} & \textbf{\# Nodes} & \textbf{\# Edges} & \textbf{\# Features} \\
        \midrule
        Cora & 2,708 & 5,429 & 1,433 \\
        Citeseer & 3,327 & 4,732 & 3,703 \\
        Pubmed & 19,717 & 44,338 & 500 \\
        PPI & 56,944 & 818,716 & 50 \\
        \bottomrule
    \end{tabular}
    \caption{Dataset statistics after pre-processing}
    \label{tab:dataset_stats}

\end{table}


Table~\ref{tab:dataset_stats} summarizes the basic statistics of the four datasets after pre-processing.
Note that the number of features is different for each dataset, as is the number of classes and the sparsity of the adjacency matrix.


We used the same dataset splits and evaluation metrics as in the original paper.
For all datasets, we used early stopping with a patience of 100 epochs on both the cross-entropy loss and the micro-F1 score on the validation set, as mentioned in the original paper.
We also used dropout with a rate of 0.6 to prevent overfitting.

    \subsection{Hyperparameters}\label{subsec:hyperparameters}
    There aren't really any hyperparameters to speak of, other than those
    described in the model description section (the number of layers, L2
    regularization coefficient, etc.). All hyperparameters were therfore taken
    directly from the paper.

    \subsection{Model Implementation}\label{subsec:model-implementation}
    For starters, we tried using the GAT model that is bundled with Pytorch
Geometric. It appears this implementation is not flexible enough to exactly
follow what the paper did, but we tried to stay as close as possible, so we
called it as follows:

\begin{minted}{python}
from torch_geometric.nn import GAT
model = GAT(
    in_channels=dataset.num_features,
    out_channels=dataset.num_classes,
    hidden_channels=8,
    num_layers=2,
    heads=8,
    dropout=0.6,
    act='elu',
    act_first=True
)
\end{minted}

The `hidden\_channels' parameter tells us that the data from each node in the
input graph are projected to an 8-dimensional space via a linear transformation
(a matrix multiplication). The `act' parameter tells us that the exponential
linear unit is then applied to the transformed data.  `heads' indicates that
this is repeated 8 different times, once per ``attention head'', meaning there
are 8 separate linear transformations from the original data's space to
8-dimensional space. `dropout' indicates that dropout is applied with
parameter p = 0.6. Lastly, `num\_layers' indicates that what this paragraph
describes is repeated twice, with the output of the first later being fed into
the second layer.

As mentioned, this does not quite follow the models as described in the paper,
but this was a useful step for us to figure out how to feed the data into the
model, and more generally how to set everything up.
The Pytorch Geometric implementation of graph attention networks does not allow, for instance, to
specify a different activation function for each layer, which would be required
to exactly follow the paper's methodology.

We therefore built ~\href{https://github.com/kushagrasoni/CS598_DLH_GAT_Implementation/blob/master/code/GAT_Implementation_Notebook.ipynb}{\textit{four different
implementations of GAT}}, one using Pytorch Geometric's GATConv class, one using Pytorch Geometric's GATv2Conv class, and
two using Pytorch primitives.
This allowed us, to the best of our knowledge, to follow the paper's methodology exactly.

We say ``to the best of our knowledge'' because parts of the methodology are not well explained in the paper, and the
authors only published the code they used to run the model on Cora in addition
to the model itself.
There may therefore be details when running the model on the other three datasets where we deviate from the paper.
We are confident, however, that any such deviations are small.


    \subsection{Computational Requirements}\label{subsec:computational-requirements}
    To reproduce the results reported in the GAT paper, we ran the GAT model using
PyTorch version 1.9.0~\cite{paszke2019pytorch} on two local machines, a Macbook
and Dell laptop running Solus OS (a Linux distribution), both of which have
Intel Core i7 CPUs (2.6 GHz on the Macbook, 1.8 GHz on the Dell machine), 16GB
RAM, and Intel UHD GPUs.

On the local systems, we installed PyTorch and all necessary dependencies using
the pip package manager. Training and evaluating the GAT model on the Cora and
Citeseer datasets took approximately 20-25 seconds per 200 epochs. Training and
evaluating the GAT model on the Pubmed dataset took approximately 2 and half
minutes with 200 training epochs. Training and evaluating the GAT model on the
PPI dataset took sufficiently long that we only ran it once to completion and
at this point had not yet implemented a timing mechanism in our code. The PPI
training ran for approximately two entire work days, so roughly 16 hours.

Overall, the computational requirements for reproducing the GAT results on a
local system are moderate, but may require a GPU for faster training times,
which is especially important for the PPI dataset.


    \section{Results}\label{sec:results}
    So far we have only been able to inaccurately replicate the paper's
experiments: as mentioned earlier, the Pytorch Geometric GAT implementation is
not flexible enough to do exactly what the paper does.
However, our preliminary results are encouraging.

Our test accuracy on the Pubmed data is essentially
indistinguishable from the one reported in the paper, and the test accuracy on
the Citeseer and Cora datasets are about 5\% worse than the ones reported in
paper.

Training the model on the PPI data has proven to be significantly slower than on the other datasets.
The results we are reporting here for the PPI dataset are therefore results on the training set.
The model has run through 71 epochs and has achieved a micro-averaged F1 score of 0.374 on the training set.
The results for the other datasets are accuracy scores on the test set after 200 epochs of training.
Refer below Table~\ref{tab:results-table}.

\begin{table}
    \centering
    \begin{tabular}{@{}llll@{}}
        \toprule
        \textbf{Dataset} & \textbf{Epochs} & \textbf{Score Type} & \textbf{Results} \\
        \midrule
        Cora             & 200             & Accuracy            & \textit{0.82}    \\
        Citeseer         & 200             & Accuracy            & \textit{0.73}    \\
        Pubmed           & 200             & Accuracy            & \textit{0.79}    \\
        PPI              & 71              & F1 Score            & \textit{0.374}   \\
        \bottomrule

    \end{tabular}
    \caption{Results using torch geometric GAT library}
    \label{tab:results-table}
\end{table}

We see signs of overfitting in our experiments and believe that reducing the overfitting will improve our test accuracy further.
Specifically, on the Citeseer, Cora and Pubmed datasets, the training accuracy reaches 100\%.
The paper implements an early stopping mechanism during training, presumably to avoid this exact problem.
Unfortunately, there are no details about how the mechanism works; it is described in a single sentence.
We will attempt to implement our own such mechanism and hope it will reduce overfitting and improve test accuracy.


    \section{Discussion}\label{sec:discussion}
    Our results show that, despite some small details being unclear from the
paper, the paper's results can be closely replicated fairly
straightforwardly. Having access to an NVIDIA GPU would likely have made
it possible to experiment more with the PPI dataset. Having a stronger
background in linear algebra would have enabled us to implement the model
using Pytorch's primitives more easily, and to optimize the code to make it
faster.

Our ablation results are somewhat surprising. Both single-layer models
performed worse than the architectures used in the paper, which is not
surprising, but one of the single-layer models appears to have performed
better than the other, which is somewhat surprising. The only difference
between the two is that the second used two activation functions: the
exponential linear unit followed by softmax. The first, on the other hand, used
only softmax.

The three-layer model performed better than the single-layer models, which is
unsurprising, but it did not appear to perform better than the two-layer models
from the paper, which is somewhat surprising. This perhaps suggests that tuning
the hyperparameters of the model is just as important as adding more parameters
via an extra layer.

As mentioned previously, the PPI dataset turned out to be sufficiently
large to make experimenting with it prohibitively slow, so we were unable
to experiment with it as thoroughly as with the other three datasets. We
are unsure whether getting our code to run on a GPU would change this.

\subsection{What was easy}\label{subsec:what-was-easy}
There were a few tasks which took less effort than the others:
\begin{itemize}
    \item Reading and understand the purpose of the paper was easy. Most of the important sections,
    namely the layer structure, the early stopping mechanism, and the models parameters, were explained nicely.
    \item Finding and accessing the datasets was easy and was made easier by torch\_geometirc's Planetoid library.
    \item Utilizing the packaged open-source GAT library to train and test the various datasets was easy.
    All it requires is providing the exact same input parameters which were used by the original authors.
\end{itemize}


\subsection{What was difficult}\label{subsec:what-was-difficult}
\begin{itemize}
    \item Complex architecture: Understanding the architecture was definitely one of the most difficult tasks, let alone implement it.
    The GAT model has a complex architecture, involving multiple layers and attention mechanisms, which was difficult to
    implement.
    \item Memory constraints: The GAT model can require a large amount of memory,
    particularly when processing large graphs or using large batch sizes like that of PPI, which can make it challenging
    to train on standard hardware.
    \item GAT implementation using Pytorch: Implementing the GAT using Pytorch was especially difficult. It easy to make subtle mistakes while wiring up the various parts of the model in Pytorch, and it is difficult to debug what is happening since the heavy lifting in Pytorch happens in C++ rather than Python, which makes it impossible to step through in Python debugger.

\end{itemize}
\subsection{Recommendations for reproducibility}\label{subsec:recommendations}
\begin{itemize}
    \item Owning an NVIDIA GPU and running the models on the GPU might help to train the models faster and to be able to handle larger datasets.
    \item Having a solid understanding of linear algebra is very important. It will help understand the model quickly, it will help translate the math behind the paper into code, and it will help to optimize the operations of the model.
    \item Knowing good debugging techniques and having experience debugging neural network models is tremendously valuable.
\end{itemize}


    \section{Communication with original authors}
    We did not communicate with the paper's authors at all.

    \bibliographystyle{unsrt} %Reference style.
    \bibliography{references}

\end{document}

