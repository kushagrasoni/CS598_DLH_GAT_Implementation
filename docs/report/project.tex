\documentclass{article}
\usepackage{minted}
\usepackage{setspace}
\usepackage{graphicx}
\usepackage{float}
\usepackage{amsmath}
\usepackage{amsfonts}
\usepackage{amssymb}
\usepackage{microtype}
\usepackage[hyperref]{project}
\usepackage{pythonhighlight}
\usepackage{booktabs,chemformula}
\renewcommand{\UrlFont}{\ttfamily\small}

\aclfinalcopy


\newcommand\BibTeX{B\textsc{ib}\TeX}

\title{Spring 2023 CS598 DL4H: Graph Attention Networks Reproducibility Project}

\author
{Valle-Mena, Ricardo Ruy and Soni, Kushagra \\
    \texttt{\{rrv4, soni14\}@illinois.edu}
    \\[2em]
    Group ID: 179\\
    Paper ID: 7\\
    Code link: \url{https://github.com/kushagrasoni/CS598_DLH_GAT_Implementation}
    \\[2em]
}

\begin{document}
    \maketitle
\section{Introduction}\label{sec:introduction}
    Graph Attention Networks (GATs)~\cite{velickovic2018graph} are a type of neural network architecture designed for
    processing graph-structured data.
    They aim to overcome the limitations of previous architectures by being able to operate on arbitrarily structured graphs in a parallelizable manner.
    GATs use a shared masked self-attention mechanism to assign weights to each node's neighbors and combine their features, resulting in a new set of features for the node in question.
    GATs have shown promising results in various graph-based tasks, including node classification and link prediction.

    \section{Scope of Reproducibility}\label{sec:scope-of-reproducibility}

    \subsection{Background and Problem Statement}\label{sec:background-and-problem-statement}
    Before Graph Attention Networks (GATs), several neural network architectures were proposed to process graph-structured data, including Graph Convolutional Networks (GCNs), GraphSAGE, and DeepWalk. However, these architectures had one or more of the following limitations:

\begin{itemize}
    \item Inability to operate on arbitrarily structured graphs: GCNs, for example, are limited to processing homogeneous graphs where all nodes have the same features.

    \item Need to sample from input graphs: some architectures, like GraphSAGE, require sampling from input graphs to create mini-batches, which can lead to information loss.

    \item Learning separate weight matrices for different node degrees: GCNs use different weight matrices for nodes with different degrees, making the model less scalable.

    \item Inability to parallelize across nodes: traditional GCNs perform summation operations over a node's neighbors, which cannot be parallelized.
\end{itemize}

Overall, prior architectures faced difficulties in handling the complexity and heterogeneity of real-world graphs, which motivated the development of Graph Attention Networks.

GATs aim to be able to operate on arbitrarily structured graphs in a manner that is parallelizable across nodes in the graph, thus having none of the
limitations mentioned previously.
GATs use a masked shared self-attention mechanism.
The mask ensures that, for a given node, only features from first-degree neighbours are taken into consideration.
The self-attention mechanism allows the model to assign arbitrary weights to each of a given
node's neighbours, which then allows the neighbours' features to be
combined, which results in a new set of features for the node in question.
The paper only mentions using the resulting features for classification
tasks, but it should also be possible to use these features for regression
tasks.

    \subsection{Objectives}\label{sec:objectives}
    Our goal was to reproduce the results reported in the original paper, which used the Cora, Citeseer, Pubmed, and Protein-protein interaction datasets.
    Specifically, we aimed for:
    \begin{itemize}
        \item Classification accuracies of approximately 83.0\%, 72.5\%, and 79\% in the Cora, Citeseer, and Pubmed
        datasets, respectively,
        \item Micro-averaged F1 score of approximately 0.973 in the protein-protein interaction dataset.
    \end{itemize}

    Furthermore, we conducted the following ablation studies:
    \begin{itemize}
        \item Using one, two, and three layer models for the Citeseer, Cora, and Pubmed datasets
	\item Not using dropout
	\item Not using L2 regularization
	\item Not using dropout or L2 regularization
    \end{itemize}
    We hypothesized that one-layer models would perform worse than two-layer models, which would in turn perform worse than three-layer models.
    We also hypothesized that not using dropout would perform worse than using dropout, that not using L2 regularization would perform worse than using L2 regularization, and that using neither dropout nor L2 regularization would perform worse of all.

    \section{Methodology}\label{sec:methodology}
    We tried a total of 5 GAT implementations, four of which were our own, and
    one of which is from the Pytorch Geometric library. One of our
    implementations was strongly influenced by the paper's authors'
    implementation ~\cite{petarvgatgithub}.

    All datasets from the paper are publicly available in multiple locations,
    including the Pytorch Geometric library. We have been running our
    experiments locally. Despite the datasets being relatively small, the PPI
    dataset turned out to be sufficiently large to take a prohibitively long
    time to run. We therefore do not provide the same amount of results for the
    PPI dataset as for the other three.

    \subsection{Model Description}\label{subsec:model-description}
    There are several variants of the same model used in the paper.

For the Cora and Citeseer datasets, two-layer GAT models were used.  The first
layer has 8 attention heads, projects the input graph's features to an
8-dimensional feature space, and uses an exponential linear unit (ELU) as its
activation function.  The second layer has a single attention head, projects
the data to a C-dimensional feature space, where C is the number of classes in
the dataset, and uses a softmax activation function.  L2 regularization is
applied with lambda = 0.0005, and dropout is applied with p = 0.6.

Should we include a more detailed description? Should we explain what each
attention head does, etc.?

For the Pubmed data, the architecture is mostly the same. However, the second
layer has 8 attention heads like the first layer, and the L2 regularization
uses a coefficient of 0.001 instead of 0.0005.

For the protein-protein interaction data, a three-layer model is used. The
first two layers have 4 attention heads, project their input data to a
256-dimensional feature space, and use an ELU activation function. The third
layer has 6 attention heads, projects its input data to a 121-dimensional
feature space, averages all 121 dimensions, and applies a softmax activation
function.


    \subsection{Data Description}\label{subsec:data-description}
    In this project, we aim to reproduce the results reported in the Graph Attention Networks (GAT) paper~\cite{velickovic2018graph}.
The paper evaluates the performance of GAT on four benchmark datasets for node classification: Cora, Citeseer, Pubmed, and Protein-Protein Interaction (PPI).

\subsubsection{Cora Dataset}\label{subsubsec:cora-dataset}

The Cora dataset is a citation network consisting of 2,708 scientific publications, where each publication is classified into one of seven classes: Case-Based Reasoning (CBR), Genetic Algorithms (GA), Neural Networks (NN), Probabilistic Methods (PM), Reinforcement Learning (RL), Rule Learning (RL), and Theory (TH).
The dataset includes a binary bag-of-words feature vector for each publication, representing the presence or absence of certain words in the document.
The network is formed by connecting publications that have a common citation.

Following the preprocessing steps described in the GAT paper, we use the same dataset split as provided by~\cite{sen2008collective}, which consists of 140 labeled nodes for training, 500 nodes for validation, and 1,000 nodes for testing.
We also use the same feature vectors for each node and the same network structure.

To represent the graph, we construct a sparse adjacency matrix $A$, where $A_{ij}=1$ if there is a citation from node $i$ to node $j$, and $A_{ij}=0$ otherwise.
We also add self-loops to the adjacency matrix, i.e., $A_{ii}=1$ for all nodes $i$.
We normalize the adjacency matrix by the degree matrix $D$, where $D_{ii}$ is the degree of node $i$, to obtain the normalized adjacency matrix $\tilde{A}=D^{-1/2}AD^{-1/2}$.
We use the same preprocessed data as provided in the original GAT paper.

To evaluate the performance of the GAT model on the Cora dataset, we use the same evaluation metrics as reported in the original paper.
Specifically, we report the micro-F1 score on the test set, which measures the harmonic mean of precision and recall across all classes.
We also report the training and validation loss during training, as well as the accuracy and micro-F1 score on the validation set, to monitor the model's performance during training.

\subsubsection{Citeseer Dataset}\label{subsubsec:citeseer-dataset}

The Citeseer dataset is a citation network consisting of 3,312 scientific publications, where each publication is classified into one of six classes: Agents, AI, DB, IR, ML, and HCI. The dataset includes a binary bag-of-words feature vector for each publication, representing the presence or absence of certain words in the document.
The network is formed by connecting publications that have a common citation.

Following the preprocessing steps described in the GAT paper, we use the same dataset split as provided by~\cite{sen2008collective}, which consists of 120 labeled nodes for training, 500 nodes for validation, and 1,000 nodes for testing.
We also use the same feature vectors for each node and the same network structure.

To represent the graph, we construct a sparse adjacency matrix $A$, where $A_{ij}=1$ if there is a citation from node $i$ to node $j$, and $A_{ij}=0$ otherwise.
We also add self-loops to the adjacency matrix, i.e., $A_{ii}=1$

    \subsection{Hyperparameters}\label{subsec:hyperparameters}
    There aren't really any hyperparameters to speak of, other than those
    described in the model description section (the number of layers, L2
    regularization coefficient, etc.). All hyperparameters were therfore taken
    directly from the paper.

    \subsection{Model Implementation}\label{subsec:model-implementation}
    ADD LINK TO REPO

For starters, we tried using the GAT model that is bundled with Pytorch
Geometric.  It appears this implementation is not flexible enough to exactly
follow what the paper did, but we tried to stay as close as possible, so we
called it as follows:

\begin{minted}{python}
from torch_geometric.nn import GAT
model = GAT(
    in_channels=dataset.num_features,
    out_channels=dataset.num_classes,
    hidden_channels=8,
    num_layers=2,
    heads=8,
    dropout=0.6,
    act='elu',
    act_first=True
)
\end{minted}

The `hidden\_channels' parameter tells us that the data from each node in the
input graph are projected to an 8-dimensional space via a linear transformation
(a matrix multiplication).  The `act' parameter tells us that the exponential
linear unit is then applied to the transformed data.  `heads' indicates that
this is repeated 8 different times, once per ``attention head'', meaning there
are 8 separate linear transformations from the original data's space to
8-dimensional space.  `dropout' indicates that dropout is applied with
parameter p = 0.6. Lastly, `num\_layers' indicates that what this paragraph
describes is repeated twice, with the output of the first later being fed into
the second layer.

As mentioned, this does not quite follow the models as described in the paper,
but this was a useful step for us to figure out how to feed the data into the
model, and more generally how to set everything up. The Pytorch Geometric
implementation of graph attention networks does not allow, for instance, to
specify a different activation function for each layer, which would be required
to exactly follow the paper's methodology.

We therefore built four different implementations of GAT, one using Pytorch
Geometric's GATConv class, one using Pytorch Geometric's GATv2Conv class, and
two using Pytorch primitives. This allowed us, to the best of our knowledge, to
follow the paper's methodology exactly. We say "to the best of our knowledge"
because parts of the methodology are not well explained in the paper, and the
authors only published the code they used to run the model on Cora in addition
to the model itself. There may therefore be details when running the model on
the other three datasets where we deviate from the paper. We are confident,
however, that any such deviations are small.


    \subsection{Computational Requirements}\label{subsec:computational-requirements}
    To reproduce the results reported in the GAT paper, we implemented the GAT model using PyTorch version 1.9.0~\cite{paszke2019pytorch} on two platforms:
\begin{itemize}
    \item two local systems, a macbook and a Windows both with an Intel Core i7 CPU (2.6 GHz and 1.8 GHz), 16GB RAM,
    and Intell UHD GPU;
    \item Google Colab's free GPU instance with 12GB of RAM\@.
\end{itemize}

On the local system, we installed PyTorch and all necessary dependencies using the pip package manager.
At part of next stage of the project, we will try to use the CUDA Toolkit to enable GPU acceleration.
Training and evaluating the GAT model on the Cora and Citeseer datasets took approximately 20-25 seconds per 200
epochs, depending on the batch size and learning rate used.

On Google Colab, we used the provided Jupyter notebook environment.
We installed PyTorch and all necessary dependencies within the notebook.
We also used Google's free Compute instance with around 12GB of RAM, which allowed us to train the GAT model on larger
datasets such as Pubmed and PPI\@.
Training the GAT model on the Pubmed dataset took approximately 20 seconds per 200 epochs, while training on the PPI
dataset took approximately 30 minutes per epoch which is also not consistent due to large memory requirement.

Overall, the computational requirements for reproducing the GAT results on a local system are moderate, but may require a GPU for faster training times.
Google Colab provides a convenient and free platform for reproducing the results, but training times may be longer due to the limited RAM and shared GPU resources.


    \section{Results}\label{sec:results}
    As part of the experimentation, we utilized 3 different open-source GAT model versions:
\begin{itemize}
    \item torch\_geometric.nn.GAT
    \item torch\_geometric.nn.GATConv
    \item torch\_geometric.nn.GATv2Conv
\end{itemize}

We were able to closely replicate the paper's
experiments via 3 different: as mentioned earlier, the Pytorch Geometric GAT implementation is
not flexible enough to do exactly what the paper does.
However, our preliminary results are encouraging.

Our test accuracy on the Pubmed data is essentially
indistinguishable from the one reported in the paper, and the test accuracy on
the Citeseer and Cora datasets are about 5\% worse than the ones reported in
paper.

Training the model on the PPI data has proven to be significantly slower than on the other datasets.
The results we are reporting here for the PPI dataset are therefore results on the training set.
The model has run through 71 epochs and has achieved a micro-averaged F1 score of 0.374 on the training set.
The results for the other datasets are accuracy scores on the test set after 200 epochs of training.
Refer below Table~\ref{tab:results-table}.

\begin{table}
    \centering
    \begin{tabular}{@{}llll@{}}
        \toprule
        \textbf{Dataset} & \textbf{GAT Type} & \textbf{Mean} & \textbf{Std. Dev.}\\
        \midrule
        Cora             &  {GAT}            &  {78.99}\%      & {1.13}\%             \\
        Cora             &  {GATConv}        &  {78.86}\%      & {1.15}\%             \\
        Cora             &  {GATv2Conv}      &  {78.53}\%      & {1.04}\%             \\
        Citeseer         &  {GAT}            &  {67.01}\%      & {1.43}\%             \\
        Citeseer         &  {GATConv}        &  {66.83}\%      & {1.41}\%             \\
        Citeseer         &  {GATv2Conv}      &  {66.65}\%      & {1.47}\%             \\
        Pubmed           &  {GAT}            &  {77.73}\%      & {1.11}\%             \\
        Pubmed           &  {GATConv}        &  {77.38}\%      & {1.32}\%             \\
        Pubmed           &  {GATv2Conv}      &  {77.73}\%      & {1.22}\%             \\
        \bottomrule

    \end{tabular}
    \caption{Results using various GAT implementations. Each dataset was executed with each GAT type 50 times
    and the Mean and Standard deviation results was taken}
    \label{tab:results-table}
\end{table}

We also implemented early stopping strategy which improved the test accuracy results to some extent.

But, since the above GAT models libraries weren't able to replicate the results 100\% with those in the original paper,
we tried to write our own GAT model implementation from scratch using torch.nn library.
Unfortunately, this attempt couldn't go as planned as we


    \section{Discussion}\label{sec:discussion}
    Our results show that, despite some small details being unclear from the
paper, the paper's results can be closely replicated fairly
straightforwardly. Having access to an NVIDIA GPU would likely have made
it possible to experiment more with the PPI dataset. Having a stronger
background in linear algebra would have enabled us to implement the model
using Pytorch's primitives more easily, and to optimize the code to make it
faster.

DISCUSS ABLATION RESULTS

As mentioned previously, the PPI dataset turned out to be sufficiently
large to make experimenting with it prohibitively slow, so we were unable
to experiment with it as thoroughly as with the other three datasets. We
are unsure whether getting our code to run on a GPU would change this.

\subsection{What was easy}\label{subsec:what-was-easy}
There were couple of tasks which took less efforts than the others:
\begin{itemize}
    \item Reading and understand the purpose of the paper was easy. Most of the important sections
    namely, layers' structure, early stopping, implementation parameters, were explained nicely.
    \item Finding and accessing the datasets was easy and was made easier by torch\_geometirc's Planetoid library.
    \item Utilizing the packaged open-source GAT library to training and test the various datasets was easy.
    All it requires is providing the exact same input parameters which were used by the original authors.
\end{itemize}


\subsection{What was difficult}\label{subsec:what-was-difficult}
\begin{itemize}
    \item Complex architecture: Understanding the architecture was definitely one of the most difficult task, let alone implement it.
    The GAT model has a complex architecture, involving multiple layers and attention mechanisms, which was difficult to
    implement.
    \item Memory constraints: The GAT model can require a large amount of memory,
    particularly when processing large graphs or using large batch sizes like that of PPI, which can make it challenging
    to train on standard hardware.
    \item GAT implementation using pytorch: Implementing the GAT using pytorch was specifically difficult when it
    comes to replicate the results in the original paper. TBU
\end{itemize}
\subsection{Recommendations for reproducibility}\label{subsec:recommendations}
\begin{itemize}
    \item Owning an NVIDIA GPU and running the models might help in running the datasets with large number of graph
    nodes.
    \item Better understanding of linear algebra will come a long way in understanding the complexity of the paper and
    being able to look at the equations in the paper and optimize them.
    \item Better debugging techniques for models and code might help a lot in saving time when trying to figure out
    issues in the implemented models.
\end{itemize}

    \section{Communication with original authors}
    We did not communicate with the paper's authors at all.

    \bibliographystyle{unsrt} %Reference style.
    \bibliography{references}

\end{document}

