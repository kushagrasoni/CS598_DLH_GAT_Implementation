\subsubsection{Architecture}
There are several variants of the same model used in the paper.
For the Cora and Citeseer datasets, two-layer GAT models were used. The first
layer has 8 attention heads, projects the input graph's features to an
8-dimensional feature space, and uses an exponential linear unit (ELU) as its
activation function. The second layer has a single attention head, projects
the data to a C-dimensional feature space, where C is the number of classes in
the dataset, and uses a softmax activation function.  L2 regularization is
applied with lambda = 0.0005, and dropout is applied with p = 0.6.

For the Pubmed data, the architecture is mostly the same. However, the second
layer has 8 attention heads like the first layer, and the L2 regularization
uses a coefficient of 0.001 instead of 0.0005.

For the protein-protein interaction data, a three-layer model is used. The
first two layers have 4 attention heads, project their input data to a
256-dimensional feature space, and use an ELU activation function. The third
layer has 6 attention heads, projects its input data to a 121-dimensional
feature space, averages all 121 dimensions, and applies a softmax activation
function.

We employed the same early stopping strategy employed by the paper's authors,
with a patience of 100 epochs, during all training.

\subsubsection{Learning Objectives}
The learning objectives for the datasets are as follows:
\begin{itemize}
    \item Cora: To classify academic papers into one of seven classes based on their content and the citations between papers.
    \item Citeseer: Given a graph where nodes represent research papers and edges represent citation links between
    papers, the model is trained to predict the subject area of each paper based on its citation links and the text
    of its title and abstract.
    The dataset contains 3,327 papers, each belonging to one of six subject areas:
    ``Agents'', ``AI'', ``DB'', ``IR'', ``ML'', or ``HCI''.
    The objective is to correctly classify the papers into their respective subject areas
    \item Pubmed: The goal is to predict the category of a scientific publication based on the citation network of papers.
    There are three possible categories: diabetes mellitus, cardiovascular diseases, and neoplasms
    \item PPI: Given a graph where nodes represent proteins and edges represent interactions between proteins,
    the task is to predict whether proteins have each of 121 possible characteristics.
\end{itemize}
