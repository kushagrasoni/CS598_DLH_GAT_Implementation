We used the same datasets as in the GAT paper: Cora, Citeseer, Pubmed, and Protein-Protein Interaction (PPI).
The Cora, Citeseer and Pubmed datasets were originally introduced in~\cite{sen2008collective} and the PPI dataset was
introduced in~\cite{hamilton2017inductive}.

Table~\ref{tab:dataset_stats} summarizes the basic statistics of the four datasets.
Note that the number of features is different for each dataset, as is the number of classes and the sparsity of the adjacency matrix.

We used the same dataset splits and evaluation metrics as in the original paper.

\subsubsection{Cora}
The Cora dataset consists of 2,708 scientific publications classified into one of seven categories.
The citations between publications form a graph, where each node represents a publication and each edge represents a citation.

\subsubsection{Citeseer}
The Citeseer dataset consists of 3,327 scientific publications classified into one of six categories.
Similarly, the citations between publications form a graph.

\subsubsection{Pubmed}
The Pubmed dataset consists of 19,717 scientific publications from the PubMed database, where each publication is associated with one or more MeSH (Medical Subject Headings) terms.
The graph is constructed using the citation links between publications and each node represents a publication.
The task is to predict the MeSH terms associated with each publication.

\subsubsection{PPI}
In the PPI dataset, there are multiple graphs, where each graph represents a tissue and each node in the graph represents a protein.
The goal of the task is to predict the biological function labels of the proteins in a previously unseen tissue graph.
There are 121 possible labels that a protein node can have, and the task is to predict all of the labels for each protein node in the test graphs.
The dataset is divided into 20 training graphs, 2 validation graphs, and 2 test graphs.

\begin{table}
    \centering
    \begin{tabular}{@{}llllll@{}}
        \toprule
        \textbf{Dataset}&\textbf{Nodes}&\textbf{Edges}&\textbf{Features}&\textbf{Classes}\\
        \midrule
        Cora     & 2,708   & 5,429   & 1,433 & 7 \\
        Citeseer & 3,327   & 4,732   & 3,703 & 6 \\
        Pubmed   & 19,717  & 44,338  & 500   & 3 \\
        PPI      & 56,944  & 818,716 & 50    & 20 \\
        \bottomrule
    \end{tabular}
    \caption{Dataset statistics; Features represents Features/Node}
    \label{tab:dataset_stats}

\end{table}



