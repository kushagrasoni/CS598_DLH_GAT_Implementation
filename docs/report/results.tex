As part of the experimentation, we utilized 3 different open-source GAT model versions:
\begin{itemize}
    \item torch\_geometric.nn.GAT
    \item torch\_geometric.nn.GATConv
    \item torch\_geometric.nn.GATv2Conv
\end{itemize}

We were able to closely replicate the paper's
experiments via 3 different: as mentioned earlier, the Pytorch Geometric GAT implementation is
not flexible enough to do exactly what the paper does.
However, our preliminary results are encouraging.

Our test accuracy on the Pubmed data is essentially
indistinguishable from the one reported in the paper, and the test accuracy on
the Citeseer and Cora datasets are about 5\% worse than the ones reported in
paper.

Training the model on the PPI data has proven to be significantly slower than on the other datasets.
The results we are reporting here for the PPI dataset are therefore results on the training set.
The model has run through 71 epochs and has achieved a micro-averaged F1 score of 0.374 on the training set.
The results for the other datasets are accuracy scores on the test set after 200 epochs of training.
Refer below Table~\ref{tab:results-table}.

\begin{table}
    \centering
    \begin{tabular}{@{}llll@{}}
        \toprule
        \textbf{Dataset} & \textbf{GAT Type} & \textbf{Mean} & \textbf{Std. Dev.}\\
        \midrule
        Cora             &  {GAT}            &  {78.99}\%      & {1.13}\%             \\
        Cora             &  {GATConv}        &  {78.86}\%      & {1.15}\%             \\
        Cora             &  {GATv2Conv}      &  {78.53}\%      & {1.04}\%             \\
        Citeseer         &  {GAT}            &  {67.01}\%      & {1.43}\%             \\
        Citeseer         &  {GATConv}        &  {66.83}\%      & {1.41}\%             \\
        Citeseer         &  {GATv2Conv}      &  {66.65}\%      & {1.47}\%             \\
        Pubmed           &  {GAT}            &  {77.73}\%      & {1.11}\%             \\
        Pubmed           &  {GATConv}        &  {77.38}\%      & {1.32}\%             \\
        Pubmed           &  {GATv2Conv}      &  {77.73}\%      & {1.22}\%             \\
        \bottomrule

    \end{tabular}
    \caption{Results using various GAT implementations. Each dataset was executed with each GAT type 50 times
    and the Mean and Standard deviation results was taken}
    \label{tab:results-table}
\end{table}

We also implemented early stopping strategy which improved the test accuracy results to some extent.

But, since the above GAT models libraries weren't able to replicate the results 100\% with those in the original paper,
we tried to write our own GAT model implementation from scratch using torch.nn library.
Unfortunately, this attempt couldn't go as planned as we
