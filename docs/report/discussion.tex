Our results show that, despite some small details being unclear from the
paper, the paper's results can be closely replicated fairly
straightforwardly. Having access to an NVIDIA GPU would likely have made
it possible to experiment more with the PPI dataset. Having a stronger
background in linear algebra would have enabled us to implement the model
using Pytorch's primitives more easily, and to optimize the code to make it
faster.

Our ablation results are somewhat surprising. Both single-layer models
performed worse than the architectures used in the paper, which is not
surprising, but one of the single-layer models appears to have performed
better than the other, which is somewhat surprising. The only difference
between the two is that the second used two activation functions: the
exponential linear unit followed by softmax. The first, on the other hand, used
only softmax.

The three-layer model performed better than the single-layer models, which is
unsurprising, but it did not appear to perform better than the two-layer models
from the paper, which is somewhat surprising. This perhaps suggests that tuning
the hyperparameters of the model is just as important as adding more parameters
via an extra layer.

As mentioned previously, the PPI dataset turned out to be sufficiently
large to make experimenting with it prohibitively slow, so we were unable
to experiment with it as thoroughly as with the other three datasets. We
are unsure whether getting our code to run on a GPU would change this.

\subsection{What was easy}\label{subsec:what-was-easy}
There were a few tasks which took less effort than the others:
\begin{itemize}
    \item Reading and understand the purpose of the paper was easy. Most of the important sections,
    namely the layer structure, the early stopping mechanism, and the models parameters, were explained nicely.
    \item Finding and accessing the datasets was easy and was made easier by torch\_geometirc's Planetoid library.
    \item Utilizing the packaged open-source GAT library to train and test the various datasets was easy.
    All it requires is providing the exact same input parameters which were used by the original authors.
\end{itemize}


\subsection{What was difficult}\label{subsec:what-was-difficult}
\begin{itemize}
    \item Complex architecture: Understanding the architecture was definitely one of the most difficult tasks, let alone implement it.
    The GAT model has a complex architecture, involving multiple layers and attention mechanisms, which was difficult to
    implement.
    \item Memory constraints: The GAT model can require a large amount of memory,
    particularly when processing large graphs or using large batch sizes like that of PPI, which can make it challenging
    to train on standard hardware.
    \item GAT implementation using Pytorch: Implementing GAT using Pytorch was especially difficult. It is easy to make subtle mistakes while wiring up the various parts of the model in Pytorch, and it is difficult to debug what is happening since the heavy lifting in Pytorch happens in C++ rather than Python, which makes it impossible to step through in Python debugger.

\end{itemize}
\subsection{Recommendations for reproducibility}\label{subsec:recommendations}
\begin{itemize}
    \item Owning an NVIDIA GPU and running the models on the GPU might help to train the models faster and to be able to handle larger datasets.
    \item Having a solid understanding of linear algebra is very important. It will help understand the model quickly, it will help translate the math behind the paper into code, and it will help to optimize the operations of the model.
    \item Knowing good debugging techniques and having experience debugging neural network models is tremendously valuable.
\end{itemize}
