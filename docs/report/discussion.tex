Our results show that, despite some small details being unclear from the
paper, the paper's results can be closely replicated fairly
straightforwardly. Having access to an NVIDIA GPU would likely have made
it possible to experiment more with the PPI dataset. Having a stronger
background in linear algebra would have enabled us to implement the model
using Pytorch's primitives more easily, and to optimize the code to make it
faster.

DISCUSS ABLATION RESULTS

As mentioned previously, the PPI dataset turned out to be sufficiently
large to make experimenting with it prohibitively slow, so we were unable
to experiment with it as thoroughly as with the other three datasets. We
are unsure whether getting our code to run on a GPU would change this.

\subsection{What was easy}\label{subsec:what-was-easy}
There were couple of tasks which took less efforts than the others:
\begin{itemize}
    \item Reading and understand the purpose of the paper was easy. Most of the important sections
    namely, layers' structure, early stopping, implementation parameters, were explained nicely.
    \item Finding and accessing the datasets was easy and was made easier by torch\_geometirc's Planetoid library.
    \item Utilizing the packaged open-source GAT library to training and test the various datasets was easy.
    All it requires is providing the exact same input parameters which were used by the original authors.
\end{itemize}


\subsection{What was difficult}\label{subsec:what-was-difficult}
\begin{itemize}
    \item Complex architecture: Understanding the architecture was definitely one of the most difficult task, let alone implement it.
    The GAT model has a complex architecture, involving multiple layers and attention mechanisms, which was difficult to
    implement.
    \item Memory constraints: The GAT model can require a large amount of memory,
    particularly when processing large graphs or using large batch sizes like that of PPI, which can make it challenging
    to train on standard hardware.
    \item GAT implementation using pytorch: Implementing the GAT using pytorch was specifically difficult when it
    comes to replicate the results in the original paper. TBU
\end{itemize}
\subsection{Recommendations for reproducibility}\label{subsec:recommendations}
\begin{itemize}
    \item Owning an NVIDIA GPU and running the models might help in running the datasets with large number of graph
    nodes.
    \item Better understanding of linear algebra will come a long way in understanding the complexity of the paper and
    being able to look at the equations in the paper and optimize them.
    \item Better debugging techniques for models and code might help a lot in saving time when trying to figure out
    issues in the implemented models.
\end{itemize}