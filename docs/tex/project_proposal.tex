\documentclass{article}

\usepackage{setspace}
\usepackage{graphicx}
\usepackage{float}
\usepackage{amsmath}
\usepackage{amsfonts}
\usepackage{amssymb}

\title{Project Proposal: Implementing Graph Attention Networks}
\author{Valle, Ruy\\
\texttt{rrv4@illinois.edu}
\and
Soni, Kushagra\\
\texttt{soni14@illinois.edu}
}
\date{\today}

\begin{document}
    \maketitle

    \onehalfspacing


    \section{Introduction}\label{sec:introduction}

    Protein interactions play a crucial role in various biological processes, such as signal transduction,
    metabolism, and immune response.
    Identifying protein interactions can provide valuable insights into these processes, and can also lead to the discovery of new drug targets.
    However, experimental methods for identifying protein interactions are often time-consuming and expensive.
    In recent years, there has been increasing interest in using machine learning methods to predict protein interactions from protein sequences and structures.
    In this project, we propose to implement Graph Attention Networks (GAT) for predicting protein interactions.
    We will try to replicate the results which were received by the original authors of Graph Attention Networks \cite{velickovic2018graph}


    \section{Objectives}\label{sec:objectives}

    The main objective of this project is to develop a machine learning model that can accurately predict protein interactions.
    Specifically, we aim to achieve the following objectives:

    \begin{itemize}
        \item Develop a GAT model that can take protein sequences and structures as input and predict protein interactions.
        \item Evaluate the performance of the GAT model on benchmark datasets for protein interaction prediction.
        \item Compare the performance of the GAT model with other state-of-the-art machine learning models for protein interaction prediction.
    \end{itemize}


    \section{Methodology}\label{sec:methodology}

    The GAT model will be implemented using Python and the PyTorch deep learning library.
    The input to the model will be a graph where nodes represent amino acids and edges represent physical interactions between the amino acids.
    The graph will be constructed from the protein sequence and structure using established methods.
    The GAT model will use attention mechanisms to weight the importance of neighboring nodes when aggregating information for each node.
    The model will be trained using cross-entropy loss and optimized using the Adam optimizer.


    \section{Evaluation}\label{sec:evaluation}

    The performance of the GAT model will be evaluated on benchmark datasets for protein interaction prediction~\cite{zitnik2017predicting},
    such as the Protein-Protein Interaction Extraction (PPI) dataset that consists of graphs corresponding to different human tissues.
    The dataset contains 20 graphs for training, 2 for validation and 2 for testing.
    To construct the graphs, we will use the preprocessed data provided by ~\cite{hamilton2017inductive}.
    The evaluation metrics will include precision, recall and F1-score.
    The performance of the GAT model will be compared with other state-of-the-art machine learning models for protein interaction prediction,
    such as the Graph Convolutional Networks (GCN)~\cite{kipf2016semi}.


    \section{Timeline}\label{sec:timeline}

    \begin{itemize}
        \item Week 1-2: Literature review and data collection
        \item Week 3-4: Graph construction and feature extraction
        \item Week 5-7: GAT model implementation and optimization
        \item Week 8-9: Model evaluation and comparison with other models
        \item Week 10-11: Results analysis and report writing
        \item Week 12: Final presentation and submission
    \end{itemize}


    \section{Conclusion}\label{sec:conclusion}

    The successful implementation of the GAT model for predicting protein interactions will have important implications for the field of computational biology.
    The model could be used to predict protein interactions for new proteins and provide valuable insights into biological processes.
    The proposed project will also provide valuable experience in implementing deep learning models for graph-structured data.


    \bibliographystyle{unsrt} %Reference style.
    \bibliography{bib/references}

\end{document}

