We used the same datasets as in the GAT paper: Cora, Citeseer, Pubmed, and Protein-Protein Interaction (PPI).
The Cora, Citeseer and Pubmed datasets were originally introduced in~\cite{sen2008collective} and the PPI dataset was
introduced in~\cite{hamilton2017inductive}.

\subsubsection{Cora}
The Cora dataset consists of 2,708 scientific publications classified into one of seven categories.
The citations between publications form a graph, where each node represents a publication and each edge represents a citation.

\subsubsection{Citeseer}
The Citeseer dataset consists of 3,327 scientific publications classified into one of six categories.
Similarly, the citations between publications form a graph.

\subsubsection{Pubmed}
The Pubmed dataset consists of 19,717 scientific publications from the PubMed database, where each publication is associated with one or more MeSH (Medical Subject Headings) terms.
The graph is constructed using the citation links between publications and each node represents a publication.
The task is to predict the MeSH terms associated with each publication.

\subsubsection{PPI}
The PPI dataset consists of 24 protein-protein interaction networks, where each network is represented by a graph with nodes corresponding to proteins and edges representing interactions between them.
The task is to predict whether a pair of proteins interacts or not.

\begin{table}
    \centering
    \begin{tabular}{@{}lllll@{}}
        \toprule
        \textbf{Dataset} & \textbf{\# Nodes} & \textbf{\# Edges} & \textbf{\# Features} \\
        \midrule
        Cora & 2,708 & 5,429 & 1,433 \\
        Citeseer & 3,327 & 4,732 & 3,703 \\
        Pubmed & 19,717 & 44,338 & 500 \\
        PPI & 56,944 & 818,716 & 50 \\
        \bottomrule
    \end{tabular}
    \caption{Dataset statistics after pre-processing}
    \label{tab:dataset_stats}

\end{table}


Table~\ref{tab:dataset_stats} summarizes the basic statistics of the four datasets.
Note that the number of features is different for each dataset, as is the number of classes and the sparsity of the adjacency matrix.


We used the same dataset splits and evaluation metrics as in the original paper.
We also used dropout with a rate of 0.6 to prevent overfitting.
